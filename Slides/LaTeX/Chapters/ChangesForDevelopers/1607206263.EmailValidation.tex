% ------------------------------------------------------------------------------
% TYPO3 Version 11.0 - What's New (German Version)
%
% @license	Creative Commons BY-NC-SA 3.0
% @link		http://typo3.org/download/release-notes/whats-new/
% @language	German
% ------------------------------------------------------------------------------
% Feature | 92531 | Improved Email Validation

\begin{frame}[fragile]
	\frametitle{Änderungen für Entwickler}
	\framesubtitle{E-Mail-Validierung}

	% decrease font size for code listing
	\lstset{basicstyle=\tiny\ttfamily}

	\begin{itemize}
		\item Entwickler können die folgende Methode verwenden, um E-Mail-Adressen zu validieren:
			\newline\smaller
				\texttt{TYPO3\textbackslash
					CMS\textbackslash
					Core\textbackslash
					Utility\textbackslash
					GeneralUtility::validEmail()}\normalsize

		\item Seit TYPO3 v11.0 ist es möglich, die Validatoren zu konfigurieren und eigene Lösungen
			zu implementieren.

		\item Zum Beispiel:
\begin{lstlisting}
$GLOBALS['TYPO3_CONF_VARS']['MAIL']['validators'] = [
  \Egulias\EmailValidator\Validation\RFCValidation::class,
  \Egulias\EmailValidator\Validation\DNSCheckValidation::class
];
\end{lstlisting}

		\item Die folgenden Standardvalidierer sind verfügbar:
			\begin{itemize}\smaller
				\item \texttt{\textbackslash
					Egulias\textbackslash
					EmailValidator\textbackslash
					Validation\textbackslash
					RFCValidation} (used by default)
				\item \texttt{\textbackslash
					Egulias\textbackslash
					EmailValidator\textbackslash
					Validation\textbackslash
					DNSCheckValidation}
				\item \texttt{\textbackslash
					Egulias\textbackslash
					EmailValidator\textbackslash
					Validation\textbackslash
					SpoofCheckValidation}
				\item \texttt{\textbackslash
					Egulias\textbackslash
					EmailValidator\textbackslash
					Validation\textbackslash
					NoRFCWarningsValidation}
			\end{itemize}
	\end{itemize}

\end{frame}

% ------------------------------------------------------------------------------
