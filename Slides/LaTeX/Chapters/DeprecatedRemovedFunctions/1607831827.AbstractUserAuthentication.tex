% ------------------------------------------------------------------------------
% TYPO3 Version 11.0 - What's New (German Version)
%
% @license	Creative Commons BY-NC-SA 3.0
% @link		http://typo3.org/download/release-notes/whats-new/
% @language	German
% ------------------------------------------------------------------------------
% Breaking | 92802 | User-database-based authentication timeout field removed
% Breaking | 92801 | Removed Failed Login functionality from User Authentication object
% Breaking | 92989 | AbstractUserAuthentication->loginFailure removed
% Breaking | 92990 | AbstractUserAuthentication->svConfig removed
% Breaking | 93073 | AbstractUserAuthentication->forceSetCookie removed
% Deprecation | 93023 | Reworked session handling

\begin{frame}[fragile]
	\frametitle{Veraltete/entfernte Funktionen}
	\framesubtitle{Klasse \texttt{AbstractUserAuthentication}}

	\begin{itemize}
		\item Die folgenden Änderungen beziehen sich auf die Klasse\newline
			\small\texttt{TYPO3\textbackslash
				CMS\textbackslash
				Core\textbackslash
				Authentication\textbackslash
				AbstractUserAuthentication}\normalsize

		\item Entfernte öffentliche Eigenschaften und Methoden:

			\begin{itemize}\smaller
				\item \texttt{AbstractUserAuthentication->auth\_timeout\_field}
				\item \texttt{AbstractUserAuthentication->loginFailure}
				\item \texttt{AbstractUserAuthentication->forceSetCookie}
				\item \texttt{AbstractUserAuthentication->svConfig}
				\item \texttt{AbstractUserAuthentication->warningEmail}
				\item \texttt{AbstractUserAuthentication->warningPeriod}
				\item \texttt{AbstractUserAuthentication->warningMax}
				\item \texttt{AbstractUserAuthentication->checkLogFailures()}
			\end{itemize}\normalsize

		\item Als Teil der Umstrukturierung der Session-Handhabung wurden diese Methoden als veraltet markiert:

			\begin{itemize}\smaller
				\item \texttt{AbstractUserAuthentication->createSessionId()}
				\item \texttt{AbstractUserAuthentication->fetchUserSession()}
			\end{itemize}\normalsize


	\end{itemize}

\end{frame}

% ------------------------------------------------------------------------------
